\documentclass[12pt]{article}

\newif\ifsol
\soltrue

% Unicode compatibility
\usepackage{iftex}
\ifPDFTeX
  \usepackage[utf8]{inputenc}
  \usepackage[noTeX]{mmap}
  \usepackage[T1]{fontenc}
\fi
% XeTeX does not support mmap
\ifLuaTeX
  \usepackage{luatex85}
  \usepackage[noTeX]{mmap}
\fi

\usepackage{framed}
%\usepackage{mdframed}
\usepackage[most]{tcolorbox}
\tcbuselibrary{breakable}

\usepackage[most]{tcolorbox}

%\newtcolorbox{mybox}{
%    enhanced,              % Required for advanced break control
%    breakable,             % Allows spanning pages
%    colback=white,
%    colframe=black,
%    sharp corners,
%    boxrule=1pt,
%    % The magic happens here:
%    breakatskip=0pt,
%    bottomrule at break=0pt, % Removes bottom line on the first part
%    toprule at break=0pt,    % Removes top line on the second part
%}


\newtcolorbox{mybox}{
    enhanced,               % Required for the "at break" keys to work
    breakable,              % Allows the box to span pages
    colback=white,          % Background color
    colframe=black,         % Border color
    sharp corners,          % Square edges like fbox
    boxrule=0.5pt,          % Standard thickness
    % This removes the lines at the break point:
    bottomrule at break=0pt,
    toprule at break=0pt,
    % Optional: Adjust padding to match fbox style
    left=5pt, right=5pt, top=5pt, bottom=5pt
}

\setlength{\textheight}{9in}
\setlength{\textwidth}{7.1in}
\setlength{\evensidemargin}{-0.2in}
\setlength{\oddsidemargin}{-0.2in}
\setlength{\headsep}{30pt}
\setlength{\topmargin}{-0.3in}
\usepackage{amsthm,amsfonts,amsmath,amssymb,caption,xspace,tikz,varwidth}
\usepackage{hyperref,verbatim}
\usepackage[capitalise, noabbrev]{cleveref}
\newtheorem{theorem}{Theorem}
\theoremstyle{definition}
\newtheorem{definition}{Definition}
\theoremstyle{remark}
\newtheorem{remark}{Remark}

\definecolor{shadecolor}{RGB}{242,242,242}


% \newcounter{defnum}
% \setcounter{defnum}{0}
% \renewenvironment{definition}
%     {\par\addbigskip
%    %\color{red}%
%     \begin{shaded*}
%     \noindent
%     \textbf{Definition.\ }\ignorespaces}
%     {\end{shaded*}
%     \addbigskip}

% \newenvironment{shadeddef}
%     {\begin{shaded*}
%     \noindent
%         \begin{definition}
%         \end{definition}
%     {\end{shaded*}}

\crefname{definition}{Definition}{Definitions}
\crefname{remark}{Remark}{Remarks}



\newcommand{\addmedskip}{\addvspace{\medskipamount}}

\newcommand{\addbigskip}{\addvspace{\bigskipamount}}

\pagestyle{myheadings}
\markboth{BU CAS CS 538.}{BU CAS CS 538.} 
\newcounter{problemnum}
\setcounter{problemnum}{0}
\newenvironment{problem}
     {\addbigskip \setcounter{partnum}{0}
      \noindent\stepcounter{problemnum}\textbf{Problem
                                             \arabic{problemnum}.\ }}
     {\par\addbigskip}


\ifsol
\newenvironment{solution}
     % {\addbigskip
     %  \noindent\textbf{Solution.\ }}
     % {\par\addbigskip}
      {\par\addbigskip
   %\color{red}%
\begin{mybox}\noindent
    \textbf{Solution.\ }\ignorespaces}
    {\end{mybox}
    \addbigskip}
\else
\newenvironment{solution}
     {\comment}
     {\endcomment}
\fi

\newcounter{partnum}
\setcounter{partnum}{0}
\newenvironment{ppart}
     {\addmedskip
      \noindent\stepcounter{partnum}\textbf{(\alph{partnum})}\ }
     {\par\addbigskip}

% \newcommand{\Enc}{\mathrm{Enc}}
% \newcommand{\Dec}{\mathrm{Dec}}
\newcommand{\eqdef}{\buildrel{\scriptstyle{\mathit{def}}}\over{=}}

\newcommand{\mbmod}{\,\%\,}

% \newcommand{\indist}{  \mathrel{\vcenter{\offinterlineskip
%   \hbox{$\sim$}\vskip-.35ex\hbox{$\sim$}\vskip-.35ex\hbox{$\sim$}}}}


\usepackage{enumerate}
\usepackage{mleftright} % for '\mleft' and '\mright' macros
\newcommand{\uprob}[2]{\Pr_{#1}\mleft[\,#2\,\mright]}

% notes
\usepackage{xcolor}
\newcommand{\julia}[1]{\textcolor{red}{\sf {\bf Julia: }#1}}
\newcommand{\leo}[1]{\textcolor{blue}{\sf {\bf Leo: } #1}}

%%%%%%%%%%%%% mike rosulek tex %%%%%%%%%%%%
% %% font stuff

\urlstyle{sf}
\usepackage[T1]{fontenc}
\usepackage{libertine}
%\usepackage[libertine]{newtxmath}
\usepackage[scaled=0.8]{beramono}

\usepackage{microtype}
\usepackage{cmap}


%% colors, styles

\newlength{\myline}
\setlength{\myline}{0.7pt} % pgf "thick"

\colorlet{bg}{white}
\colorlet{fg}{black}
\colorlet{shaded}{black!10}
\colorlet{shadeborder}{black!30}
\definecolor{hlbg}{HTML}{F5F5A4} 
\colorlet{hlfg}{black}
\colorlet{commentcolor}{black!60!white}
\definecolor{errorcolor}{HTML}{a91616}
\definecolor{linkcolor}{HTML}{4b804c}
\definecolor{bitcolor}{HTML}{a91616}

\pagecolor{bg}
\color{fg}

%% highlighting

\usepackage[auto,outline]{contour}

\contourlength{1.2\myline}
\newcommand{\hl}[1]{%
    \relax\ifmmode%
        {}%
        \contour{hlbg}{\textcolor{hlfg}{${} #1 {}$}}%
        {}%
    \else%
        \contour{hlbg}{\textcolor{hlfg}{#1}}%
    \fi%
}

%% codebox stuff 

% \usepackage{pipetex}
% \pipetexcommand{perl codebox2tex.pl}

\definecolor{boxbordercolor}{HTML}{000000}
\definecolor{boxbgcolor}{HTML}{FFFFFF}

\newcommand{\codebox}[1]{%
    \begin{varwidth}{\linewidth}%
        \upshape%   no slant in definition/theorem statement!
        \begin{tabbing}%
            ~~~\=\quad\=\quad\=\quad\=\kill
            #1
        \end{tabbing}%
    \end{varwidth}%
}

\newcommand{\fcodebox}[1]{%
    \fboxsep=3pt%
    \fcolorbox{boxbordercolor}{boxbgcolor}{\codebox{#1}}%
}

\newcommand{\titlecodebox}[2]{%
    \fboxsep=0pt%
    \fcolorbox{boxbordercolor}{boxbordercolor!10!boxbgcolor}{%
        \begin{varwidth}{\linewidth}%
            \centering%
            \fboxsep=3pt%
            \colorbox{boxbordercolor!10!boxbgcolor}{#1} \\
            \colorbox{boxbgcolor}{\codebox{#2}}%
        \end{varwidth}%
    }%
}

\newcommand{\hltitlecodebox}[2]{%
    \fboxsep=3pt%
    \colorbox{hlbg}{%
        \titlecodebox{#1}{#2}%
    }%
}

\newcommand{\hlcodebox}[1]{%
    \fboxsep=3pt%
    \colorbox{hlbg}{%
        \fcodebox{#1}%
    }%
}

\newcommand{\procheader}[1]{\underline{#1}}
\newcommand{\mycomment}[1]{\textcolor{commentcolor}{\small\textsl{// #1}}}

%% library stuff and math stuff

\renewcommand{\L}{\mathcal{L}}
\newcommand{\lib}[1]{\mathcal{L}_{\textsf{\textup{#1}}}}
\newcommand{\outputs}{\Rightarrow}
\newcommand{\link}{\diamond}

\newcommand{\indist}{\approxeq}
\renewcommand{\gets}{\twoheadleftarrow}

\renewcommand{\le}{\leqslant}
\renewcommand{\leq}{\leqslant}
\renewcommand{\ge}{\geqslant}
\renewcommand{\geq}{\geqslant}

\newcommand{\pct}{\mathbin{\%}}
\renewcommand{\le}{\leqslant}
\renewcommand{\leq}{\leqslant}
\renewcommand{\ge}{\geqslant}
\renewcommand{\geq}{\geqslant}

\newcommand{\secpar}{\lambda}

%% bits

\newcommand{\bit}[1]{\ensuremath{\textcolor{bitcolor}{\texttt{\upshape #1}}}\xspace}
\newcommand{\bits}{\{\bit0,\bit1\}}

%% algorithms
\newcommand{\algorithm}[1]{\ensuremath{\textsf{\upshape#1}}\xspace}

\newcommand{\Scheme}{\Sigma}
\newcommand{\KeyGen}{\algorithm{KeyGen}}
\newcommand{\Enc}{\algorithm{Enc}}
\newcommand{\Encaps}{\algorithm{Encaps}}
\newcommand{\Sign}{\algorithm{Sign}}
\newcommand{\Dec}{\algorithm{Dec}}
\newcommand{\Decaps}{\algorithm{Decaps}}
\newcommand{\MAC}{\algorithm{MAC}}
\newcommand{\Verify}{\algorithm{Verify}}
\newcommand{\Share}{\algorithm{Share}}
\newcommand{\Reconstruct}{\algorithm{Reconstruct}}
\newcommand{\Filter}{\algorithm{Filter}}
\newcommand{\Birthday}{\algorithm{Birthday}}
\newcommand{\Start}{\algorithm{Start}}
\newcommand{\Respond}{\algorithm{Respond}}
\newcommand{\Finalize}{\algorithm{Finalize}}
\newcommand{\RSA}{\algorithm{RSA}}
\newcommand{\Commit}{\algorithm{Commit}}
\renewcommand{\Check}{\algorithm{Check}}
\newcommand{\Extract}{\algorithm{Extract}}
\newcommand{\SimTrans}{\algorithm{SimTranscript}}

\newcommand{\subname}[1]{\textsc{#1}}

%% misc crypto

\renewcommand{\L}{\mathcal{L}}
\newcommand{\Z}{\mathbb{Z}}
\newcommand{\K}{\mathcal{K}}
\newcommand{\M}{\mathcal{M}}
\newcommand{\E}{\mathcal{E}}
\newcommand{\C}{\mathcal{C}}
\newcommand{\randk}{\mathsf k}
\newcommand{\randm}{\mathsf m}
\newcommand{\randc}{\mathsf c}
\newcommand{\Adv}{\mathcal{A}}
\newcommand{\B}{\mathcal{B}}
\newcommand{\SSadv}{\textrm{SS}\textsf{adv}}

\newcommand{\msg}{M}
\newcommand{\key}{K}
\newcommand{\ctext}{C}

\newcommand{\attack}{\textsc{attack}}
\newcommand{\getrandom}{\twoheadleftarrow}




\begin{document}
\begin{center}
\Large{\textbf{CAS CS 538.  \ifsol Solutions to \fi Problem Set 5}}\\
\smallskip
\large{\textbf{Due electronically via gradescope, \textcolor{red}{Tuesday February 24}, 2026 11:59pm}}\\
\large{\textbf{Om Khadka, U51801771}}
\end{center}

\noindent

\begin{problem} (50 points)

\noindent In Discussion 5 we introduced the Signal symmetric ratchet and we showed that even if a subsequent key is leaked to the adversary, it is hard to recover the previous key. 
In this problem, we will prove something stronger: even if a subseuqent key is leaked (and the previous key has been deleted), past encryptions are still semantically secure.

Specifically, let $\E = (\Enc, \Dec)$ be a semantically secure cipher over $(\K, \M, \C)$. 
Let $G : \S \to \S \times \K$ be a secure PRG such that $G: \left\{0,1 \right\}^\ell \to \left\{0,1 \right\}^{2\ell}$. 
Define the symmetric ratchet scheme as $\Enc^1(s, m) = \Enc(G(s)_2, m)$ and $\Dec^1(s,c) = \Dec(G(s)_2, c)$. 
Here $G(s)_2$ denotes the second part of the output of $G(s)$, i.e. the key part. 
We refer you to Figure 1 of Discussion 5 for a diagram of the symmetric ratchet.

At time step $n$, participants will advance the state of the ratchet by computing $(s_n, k_n) \gets G(s_{n-1})$.
Then, the past is erased by deleting the old seed $s_{n-1}$ from memory.
The new ciphertext is computed as $c_n = \Enc(k_n, m_n)$.
 
Prove that $\E^1$ is {\it semantically secure} for the previous encryption $c_{n-1}$ even if a subsequent seed and key $(s_n, k_n)$ is leaked to the adversary.


\paragraph{Getting started.}
To get started, you should consider an adversary $\Adv$ playing a new game called the $\SSkeyleak$ game against $\E^1$.
The $\SSkeyleak$ is the same as the Semantic Security Game (Boneh + Shoup Attack Game 2.1) except that in addition to a ciphertext $c_{n-1}$, the adversary {\it also} receives a leaked seed and key $(s_n, k_n)$ where $s_n,k_n$ are pseudorandom values produced by $G^n$ with an initial random seed.

In particular, say $\Adv$ outputs two messages $m_0, m_1$.
Let Game $0$ be when $c_{n-1}=\Enc^1(k_{n-1}, m_0)$ and Game 1 be when $c_{n-1}=\Enc(k_{n-1}, m_1)$.
$\Adv$ receives the tuple $(c_{n-1}, s_n, k_n)$ and outputs a bit $\hat{b}$.
Let $W_b$ be the event that $\hat{b} = 1$ in Game $b$ of the $\SSkeyleak$ game. 
Then we define the key-leakage advantage as $\SSkeyleakadv[\Adv, \E^1] = |\Pr[W_0] - \Pr[W_1]|$.  


\paragraph{Building a hybrid argument.}
Use a hybrid argument to prove that $\E^1$ is semantically secure against leaked keys by showing that $\SSkeyleakadv[\Adv, \E^1]$ is negligible. 

\emph{Hint: you will have to introduce two hybrid games in between Game 0 and Game 1 of $\SSkeyleak$ game. Then you will give three reductions with wrapper adversaries $\B_1$, $\B_2$, and $\B_3$. In no particular order, two of these reductions yield a PRG distinguisher for $G^n$ and one of them yields an SS adversary for $\E$.}

\end{problem}

\begin{solution}
Your solution goes here
\end{solution}

\newpage

\begin{problem} (50 points at 10 each)
  Let $p > 2$ be an odd prime.
For this problem, suppose the order of $g$ is $p-1$ (which is even). Such a $g$ is called a \emph{generator} of $\Z_p^*$. Such a $g$ exists for every $p$ (we won't prove this fact; see, for example, Section 7.5 of Victor Shoup's book at \url{http://www.shoup.net/ntb/ntb-v2.pdf} for the existence proof and Section 11.1 for how to sample it efficiently).

\begin{ppart}
Let $a = g^x$ for some $x \in \Z_{p - 1}$ (the exponent works modulo $p - 1$ due to Fermat's little theorem).
We can talk about the exponent $x$ being even since $p - 1$ is even, and therefore $x + k(p - 1)$ has the same evenness as $x$ for any $k \in \Z$.

Clearly if $x$ is even, then
$a$ has a square root $g^{x/2}$ modulo $p$.  Now show the converse: if $a$ has
a square root modulo $p$, then $x$ is even. \textit{(Hint: you can rewrite any element $b\in \mathbb{Z}_p^\ast$ as $b=g^y$ for some $y \in \mathbb{Z}_{p-1}$.)}\end{ppart}
\begin{solution}
Your solution goes here
\end{solution}

It'd be nice to check if a value is a square modulo $p$ without having to explicitly know what power of $g$ it is. Luckily we have the following test: $a$ is a square iff $a^{(p-1)/2} \equiv_p 1$.

\begin{ppart}
    Prove the forward direction: if $a$ is square then $a^{(p-1)/2} \equiv_p 1$.
\end{ppart}

\begin{solution}
Your solution goes here
\end{solution}


\begin{ppart}
    Now prove the converse: if $a^{(p-1)/2} \equiv_p 1$, then $a$ is square. \textit{(Hint: Assume not and find a contradiction. Begin by writing $a$ as $g^x$.)}
\end{ppart}

\begin{solution}
Your solution goes here
\end{solution}


\noindent
We have thus shown that exactly half the values in $\Z_p^*$ have square
roots, and we know how to identify them: by raising to $(p-1)/2$.  Note
also that values that do have square roots have at least two of them: if $r \in \Z_p^*$
is a square root of $a \in \Z_p^*$, then so is $-r$, because $(-r)^2 = (-1)\cdot r\cdot(-1)\cdot r = (-1)(-1)r^2 = 1\cdot r^2 = r^2$.
The two square roots are different, because $p-r\neq r$ as $p$ is odd. Therefore, each square
cannot have more than two square roots by
a simple counting argument: if some square had more than two square roots, there wouldn't be enough square roots for all the $(p-1)/2$ squares, because just two square roots per square already takes up all the $p-1$ possible square root values in $\Z_p^*$. Thus, each square has exactly two square roots.


\begin{ppart}
Show that if $\left(g^x\right )^2\equiv_p a$, then $\left
(g^{x+(p-1)/2}\right)^2 \equiv_p a$, as well.  Show that these two square roots are distinct: that is, show that
$g^x\not \equiv_p g^{x+(p-1)/2}$.
\end{ppart}
\begin{solution}
Your solution goes here
\end{solution}

\noindent
We know from the paragraph above that $a$ has only two square roots. But we have three values that all when squared give us $a$: $g^x$, $g^{x+(p-1)/2}$,
and $-g^x$.  Thus, it must be that $g^{x+(p-1)/2}\equiv_p -g^x$.

\begin{ppart}
Given that $g^{x+(p-1)/2}\equiv_p -g^x$, show that 
that $g^{(p-1)/2}\equiv_p -1$. Now show that for any $b\in \Z_p^*$ that is a non-square,
$b^{(p-1)/2} \equiv_p -1$.
\end{ppart}

\noindent
This refines our previous test for squares from parts (b) and (c): to test if something
is a square, you raise it to $(p-1)/2$ modulo $p$ and check if the result is 1
or $-1$.  By the way, the value of $a^{(p-1)/2} \,\%\,p$ is called the Legendre symbol
of $a$ and is often written as $\left(\frac{a}{p}\right)$. The Legendre symbol can be generalized for composite $p$ and this generalization is called the Jacobi symbol (we will not cover it here).
\end{problem}

\begin{solution}
Your solution goes here
\end{solution}

\end{document}
