\documentclass[12pt]{article}

\newif\ifsol
\soltrue

% Unicode compatibility
\usepackage{iftex}
\ifPDFTeX
  \usepackage[utf8]{inputenc}
  \usepackage[noTeX]{mmap}
  \usepackage[T1]{fontenc}
\fi
% XeTeX does not support mmap
\ifLuaTeX
  \usepackage{luatex85}
  \usepackage[noTeX]{mmap}
\fi

\usepackage{framed}
%\usepackage{mdframed}
\usepackage[most]{tcolorbox}
\tcbuselibrary{breakable}

\usepackage[most]{tcolorbox}

%\newtcolorbox{mybox}{
%    enhanced,              % Required for advanced break control
%    breakable,             % Allows spanning pages
%    colback=white,
%    colframe=black,
%    sharp corners,
%    boxrule=1pt,
%    % The magic happens here:
%    breakatskip=0pt,
%    bottomrule at break=0pt, % Removes bottom line on the first part
%    toprule at break=0pt,    % Removes top line on the second part
%}


\newtcolorbox{mybox}{
    enhanced,               % Required for the "at break" keys to work
    breakable,              % Allows the box to span pages
    colback=white,          % Background color
    colframe=black,         % Border color
    sharp corners,          % Square edges like fbox
    boxrule=0.5pt,          % Standard thickness
    % This removes the lines at the break point:
    bottomrule at break=0pt,
    toprule at break=0pt,
    % Optional: Adjust padding to match fbox style
    left=5pt, right=5pt, top=5pt, bottom=5pt
}

\setlength{\textheight}{9in}
\setlength{\textwidth}{7.1in}
\setlength{\evensidemargin}{-0.2in}
\setlength{\oddsidemargin}{-0.2in}
\setlength{\headsep}{30pt}
\setlength{\topmargin}{-0.3in}
\usepackage{amsthm,amsfonts,amsmath,amssymb,caption,xspace,tikz,varwidth}
\usepackage{hyperref,verbatim}
\usepackage[capitalise, noabbrev]{cleveref}
\newtheorem{theorem}{Theorem}
\theoremstyle{definition}
\newtheorem{definition}{Definition}
\theoremstyle{remark}
\newtheorem{remark}{Remark}

\definecolor{shadecolor}{RGB}{242,242,242}


% \newcounter{defnum}
% \setcounter{defnum}{0}
% \renewenvironment{definition}
%     {\par\addbigskip
%    %\color{red}%
%     \begin{shaded*}
%     \noindent
%     \textbf{Definition.\ }\ignorespaces}
%     {\end{shaded*}
%     \addbigskip}

% \newenvironment{shadeddef}
%     {\begin{shaded*}
%     \noindent
%         \begin{definition}
%         \end{definition}
%     {\end{shaded*}}

\crefname{definition}{Definition}{Definitions}
\crefname{remark}{Remark}{Remarks}



\newcommand{\addmedskip}{\addvspace{\medskipamount}}

\newcommand{\addbigskip}{\addvspace{\bigskipamount}}

\pagestyle{myheadings}
\markboth{BU CAS CS 538.}{BU CAS CS 538.} 
\newcounter{problemnum}
\setcounter{problemnum}{0}
\newenvironment{problem}
     {\addbigskip \setcounter{partnum}{0}
      \noindent\stepcounter{problemnum}\textbf{Problem
                                             \arabic{problemnum}.\ }}
     {\par\addbigskip}


\ifsol
\newenvironment{solution}
     % {\addbigskip
     %  \noindent\textbf{Solution.\ }}
     % {\par\addbigskip}
      {\par\addbigskip
   %\color{red}%
\begin{mybox}\noindent
    \textbf{Solution.\ }\ignorespaces}
    {\end{mybox}
    \addbigskip}
\else
\newenvironment{solution}
     {\comment}
     {\endcomment}
\fi

\newcounter{partnum}
\setcounter{partnum}{0}
\newenvironment{ppart}
     {\addmedskip
      \noindent\stepcounter{partnum}\textbf{(\alph{partnum})}\ }
     {\par\addbigskip}

% \newcommand{\Enc}{\mathrm{Enc}}
% \newcommand{\Dec}{\mathrm{Dec}}
\newcommand{\eqdef}{\buildrel{\scriptstyle{\mathit{def}}}\over{=}}

\newcommand{\mbmod}{\,\%\,}

% \newcommand{\indist}{  \mathrel{\vcenter{\offinterlineskip
%   \hbox{$\sim$}\vskip-.35ex\hbox{$\sim$}\vskip-.35ex\hbox{$\sim$}}}}


\usepackage{enumerate}
\usepackage{mleftright} % for '\mleft' and '\mright' macros
\newcommand{\uprob}[2]{\Pr_{#1}\mleft[\,#2\,\mright]}

% notes
\usepackage{xcolor}
\newcommand{\julia}[1]{\textcolor{red}{\sf {\bf Julia: }#1}}
\newcommand{\leo}[1]{\textcolor{blue}{\sf {\bf Leo: } #1}}

%%%%%%%%%%%%% mike rosulek tex %%%%%%%%%%%%
% %% font stuff

\urlstyle{sf}
\usepackage[T1]{fontenc}
\usepackage{libertine}
%\usepackage[libertine]{newtxmath}
\usepackage[scaled=0.8]{beramono}

\usepackage{microtype}
\usepackage{cmap}


%% colors, styles

\newlength{\myline}
\setlength{\myline}{0.7pt} % pgf "thick"

\colorlet{bg}{white}
\colorlet{fg}{black}
\colorlet{shaded}{black!10}
\colorlet{shadeborder}{black!30}
\definecolor{hlbg}{HTML}{F5F5A4} 
\colorlet{hlfg}{black}
\colorlet{commentcolor}{black!60!white}
\definecolor{errorcolor}{HTML}{a91616}
\definecolor{linkcolor}{HTML}{4b804c}
\definecolor{bitcolor}{HTML}{a91616}

\pagecolor{bg}
\color{fg}

%% highlighting

\usepackage[auto,outline]{contour}

\contourlength{1.2\myline}
\newcommand{\hl}[1]{%
    \relax\ifmmode%
        {}%
        \contour{hlbg}{\textcolor{hlfg}{${} #1 {}$}}%
        {}%
    \else%
        \contour{hlbg}{\textcolor{hlfg}{#1}}%
    \fi%
}

%% codebox stuff 

% \usepackage{pipetex}
% \pipetexcommand{perl codebox2tex.pl}

\definecolor{boxbordercolor}{HTML}{000000}
\definecolor{boxbgcolor}{HTML}{FFFFFF}

\newcommand{\codebox}[1]{%
    \begin{varwidth}{\linewidth}%
        \upshape%   no slant in definition/theorem statement!
        \begin{tabbing}%
            ~~~\=\quad\=\quad\=\quad\=\kill
            #1
        \end{tabbing}%
    \end{varwidth}%
}

\newcommand{\fcodebox}[1]{%
    \fboxsep=3pt%
    \fcolorbox{boxbordercolor}{boxbgcolor}{\codebox{#1}}%
}

\newcommand{\titlecodebox}[2]{%
    \fboxsep=0pt%
    \fcolorbox{boxbordercolor}{boxbordercolor!10!boxbgcolor}{%
        \begin{varwidth}{\linewidth}%
            \centering%
            \fboxsep=3pt%
            \colorbox{boxbordercolor!10!boxbgcolor}{#1} \\
            \colorbox{boxbgcolor}{\codebox{#2}}%
        \end{varwidth}%
    }%
}

\newcommand{\hltitlecodebox}[2]{%
    \fboxsep=3pt%
    \colorbox{hlbg}{%
        \titlecodebox{#1}{#2}%
    }%
}

\newcommand{\hlcodebox}[1]{%
    \fboxsep=3pt%
    \colorbox{hlbg}{%
        \fcodebox{#1}%
    }%
}

\newcommand{\procheader}[1]{\underline{#1}}
\newcommand{\mycomment}[1]{\textcolor{commentcolor}{\small\textsl{// #1}}}

%% library stuff and math stuff

\renewcommand{\L}{\mathcal{L}}
\newcommand{\lib}[1]{\mathcal{L}_{\textsf{\textup{#1}}}}
\newcommand{\outputs}{\Rightarrow}
\newcommand{\link}{\diamond}

\newcommand{\indist}{\approxeq}
\renewcommand{\gets}{\twoheadleftarrow}

\renewcommand{\le}{\leqslant}
\renewcommand{\leq}{\leqslant}
\renewcommand{\ge}{\geqslant}
\renewcommand{\geq}{\geqslant}

\newcommand{\pct}{\mathbin{\%}}
\renewcommand{\le}{\leqslant}
\renewcommand{\leq}{\leqslant}
\renewcommand{\ge}{\geqslant}
\renewcommand{\geq}{\geqslant}

\newcommand{\secpar}{\lambda}

%% bits

\newcommand{\bit}[1]{\ensuremath{\textcolor{bitcolor}{\texttt{\upshape #1}}}\xspace}
\newcommand{\bits}{\{\bit0,\bit1\}}

%% algorithms
\newcommand{\algorithm}[1]{\ensuremath{\textsf{\upshape#1}}\xspace}

\newcommand{\Scheme}{\Sigma}
\newcommand{\KeyGen}{\algorithm{KeyGen}}
\newcommand{\Enc}{\algorithm{Enc}}
\newcommand{\Encaps}{\algorithm{Encaps}}
\newcommand{\Sign}{\algorithm{Sign}}
\newcommand{\Dec}{\algorithm{Dec}}
\newcommand{\Decaps}{\algorithm{Decaps}}
\newcommand{\MAC}{\algorithm{MAC}}
\newcommand{\Verify}{\algorithm{Verify}}
\newcommand{\Share}{\algorithm{Share}}
\newcommand{\Reconstruct}{\algorithm{Reconstruct}}
\newcommand{\Filter}{\algorithm{Filter}}
\newcommand{\Birthday}{\algorithm{Birthday}}
\newcommand{\Start}{\algorithm{Start}}
\newcommand{\Respond}{\algorithm{Respond}}
\newcommand{\Finalize}{\algorithm{Finalize}}
\newcommand{\RSA}{\algorithm{RSA}}
\newcommand{\Commit}{\algorithm{Commit}}
\renewcommand{\Check}{\algorithm{Check}}
\newcommand{\Extract}{\algorithm{Extract}}
\newcommand{\SimTrans}{\algorithm{SimTranscript}}

\newcommand{\subname}[1]{\textsc{#1}}

%% misc crypto

\renewcommand{\L}{\mathcal{L}}
\newcommand{\Z}{\mathbb{Z}}
\newcommand{\K}{\mathcal{K}}
\newcommand{\M}{\mathcal{M}}
\newcommand{\E}{\mathcal{E}}
\newcommand{\C}{\mathcal{C}}
\newcommand{\randk}{\mathsf k}
\newcommand{\randm}{\mathsf m}
\newcommand{\randc}{\mathsf c}
\newcommand{\Adv}{\mathcal{A}}
\newcommand{\B}{\mathcal{B}}
\newcommand{\SSadv}{\textrm{SS}\textsf{adv}}

\newcommand{\msg}{M}
\newcommand{\key}{K}
\newcommand{\ctext}{C}

\newcommand{\attack}{\textsc{attack}}
\newcommand{\getrandom}{\twoheadleftarrow}




\begin{document}
\begin{center}
\Large{\textbf{CAS CS 538.  \ifsol Solutions to \fi Problem Set 2}}\\
\smallskip
\large{\textbf{Due electronically via gradescope, Monday February 2, 2026 11:59pm
}}
\end{center}

\subsection*{Useful Definitions}

A function is negligible is if goes to zero very quickly. How quickly? Faster than $1/n$, $1/n^2$, $1/n^3$, etc. Formally, we have the following definition.

\begin{shaded*}
\begin{definition}[Negligible function]
    A function $f : \mathbb{Z}_{\ge 1} \to \mathbb{R}$ is \textbf{negligible} if for all integers $c>0$, we have \[
        \lim_{n \to \infty} f(n) \cdot n^c  = 0
    .\]
\end{definition}
\end{shaded*}

We will need negligible functions to formalize security (basically, we will want to make sure that the probability of adversarial success is negligible). See Chapter 2.3.1 of Boneh \& Shoup textbook. 

%
%\begin{definition}[Semantic security advantage]
%    For a given cipher $\E= \left(E,D \right) $ defined over $\left( \K,\M\C \right) $ and for a given adversary $\Adv$, we define two experiments, Experiment 0 and Experiment 1. For $b\in \left\{0,1 \right\}$, we define Experiment $b$ as follows:
%    \begin{itemize}
%        \item The adversary computes $m_0, m_1 \in \M$, of the same length, and sends them to the challenger.
%        \item The challenger computs $k \rfrom \K, c \rfrom E(k, m_b)$, and sends $c$ to the adversary. 
%        \item The adversary outputs a bit $\hat{b} \in \left\{ 0,1 \right\} $.
%    \end{itemize}
%    For $b \in \left\{0,1 \right\}$, let $W_b$ be the event that $\Adv$ outputs 1 in Experiment $b$. We define $\Adv$'s \textbf{semantic security advantage} with respect to $\E$ as \[
%    \SSadv[\Adv, \E] := \left| \Pr[W_0]- \Pr[W_1] \right| 
%.\]
%\end{definition}
%
%\begin{definition}[Semantic security]
%    A cipher $\E$ is \textbf{semantically secure} if for all efficient adversaries $\Adv$, the value $\SSadv[\A, \E]$ is negligible.
%    
%\end{definition}
%
\subsection*{Problems}

\begin{problem} (15 points)

Show that each of the following functions are negligible:
\begin{enumerate}[label=\alph*)]
    \item $f(n) + g(n)$, where $f(n)$ and $g(n)$ are both negligible
    \item $1000 \cdot f(n)$, where $f(n)$ is negligible
    \item $\frac{1}{2^{\sqrt{n}}}$ 
\end{enumerate}
    
\end{problem}

\begin{solution}
Your solution goes here
\end{solution}

\begin{problem} (30 points)

    You're a rising spy working your way up the ranks of national intelligence. One day you are sent a top-secret mission via a mysterious encrypted message. It's a single clue: a three-character airport code for one of the following airports in five possible countries:
    \begin{center}
    \begin{tabular}{c | c | c | c | c}
        \textbf{USA} & \textbf{Germany} & \textbf{China} & \textbf{Brazil} & \textbf{India}\\
        \hline
        BOS & MUC & PEK & GIG & DEL\\
        JFK & FRA & SZX & BSB & BOM \\
        LAX & DUS & PVG & GRU & MAA\\
        DCA & HAM & CAN & CGH & BLR\\
    \end{tabular}
    \end{center}

Unbeknownst to you, a copy of the ciphertext was intercepted by a rival attempting to learn which country you are going to. 


Consider the following security game between a challenger and an adversary. 
Let $\Scheme=(\Enc, \Dec)$ be a computational cipher with message space $\M$ equal to the set of 20 airport codes above. 
The challenger chooses a random plaintext $m\rfrom \M$ and key $k\rfrom \K$, computes $c\rfrom \Enc(k, m)$ and sends $c$ to the adversary. The adversary $\Adv$ outputs a string $s$ from the following set: \[
    \left\{ \verb|"USA"|, \verb|"Germany"|, \verb|"China"|, \verb|"Brazil"|, \verb|"India"|\right\} 
.\]
Let $\country(m)$ be a function that outputs the country string in which a given airport code is located. Let $W$ be the event that $s = \country(m)$, let $p= \Pr[W]$, and define $\countryguessadv[\Adv, \Scheme] = |p - \frac{1}{5}|$. 

Prove that if $\Scheme$ is a semantically secure cipher (per Section 2.2.2 of the textbook), then for any efficient adversary $\A$, $\countryguessadv[\A,\Scheme]$ is negligible. Notice what this implies: your rival's attempt to learn which country you're travelling to is at most negligibly better than random guessing.

\textit{To prove this, use a reduction: supposing there exists an efficient adversary with non-negligible advantage in the country-guessing game, construct an efficient adversary with non-negligible advantage in the semantic security game. }

\end{problem}

\begin{solution}
Your solution goes here
\end{solution}
 
\begin{problem} (30 points)
Let $\Scheme$ be a computational cipher with $|\K| < |\M|$.
Construct an adversary $\Adv$ against $\Scheme$ such that the running time of $\Adv$ is very reasonable (in fact, comparable to the running time of $\Enc$ and $\Dec$) and $\SSadv[\Adv, \Scheme]>0$. Note that it is okay if $\SSadv[\Adv, \Scheme]$ is very small, as long as it is positive.

You must demonstrate what the adversary does and prove that its $\SSadv$ is positive.
The adversary should \textbf{not} depend on any knowledge about $\Scheme$ that cannot be efficiently obtained (for example, the adversary doesn't know exact probabilities of different ciphertexts).

\emph{Hint: For the adversary design, use the idea from Discussion 2 Problem 2. You can't perform exhaustive search, of course, because you don't have the time; make a random guess instead. Then analyze the probability of outputting 1 in each of the two experiments. This analysis will be different from the one in discussion --- go back to your first principles of probability. Conclude that it has non-zero advantage.}

This problem justifies why any reasonable definition of semantic security must allow for at least a negligible advantage for $\Adv$ even when the running time of $\Adv$ is limited.
\end{problem}
\begin{solution}
Your solution goes here
\end{solution}

% \begin{problem} (25 points)

% For this problem, you will prove that applying an insecure cipher on top of a secure cipher can still result in a secure cipher.

% Let $(\cenc, \cdec)$ be a Caeser cipher on bytes over $\left( \K_1, \M_1, \C_1 \right) $, where $\K_1 = \left\{0,1 \right\}^8$, $\M_1 = \left\{ 0,1 \right\} ^{8L}$, and $\C_1 = \left\{0,1 \right\}^{8L}$. Concretely,  

% \begin{figure}[h!]
% \center
% \begin{subfigure}{0.4\linewidth}
% \titlecodebox{\cenc(k, m)}{
%     \> for $i =0$ to $L-1$:\\
%     \> \> $c_i := k \oplus m[8i:8(i+1)]$\\
%     \> $c := c_0 \parallel c_1 \parallel \cdots \parallel c_{L-1}$\\
%     \> return $c$
% }
% \end{subfigure}
% \begin{subfigure}{0.4\linewidth}
% \titlecodebox{\cdec(k, c)}{
%     \> for $i =0$ to $L-1$:\\
%     \> \> $m_i := k \oplus c[8i:8(i+1)]$\\
%     \> $m := m_0 \parallel m_1 \parallel \cdots \parallel m_{L-1}$\\
%     \> return $m$
% }
% \end{subfigure}
% \end{figure}

% Let $\E=(\Enc, \Dec)$ be a semantically secure cipher over $\left( \K_2,\M_2,\C_2 \right) $, where $\C_2 = \left\{0,1 \right\}^{8L}$. Define a modified encryption scheme $\E' = \left( \Enc', \Dec' \right) $ over $\left( \K_1 \times \K_2, \M_2, \C_1 \right) $, where 
% \begin{align*}
%     \Enc'((k_1, k_2) ,m) &= \cenc(k_1, \Enc(k_2, m))\\
%     \Dec'((k_1,k_2), c) &= \Dec(k_2, (\cdec(k_1, c))
% .\end{align*}
% Prove that $\E'$ is as secure as $\E$. That is, prove that for any adversary $\Adv$, there is an adversary $\B$ of similar running time such that $\SSadv[\B, \E] = \SSadv[\Adv, \E']$.
% \end{problem}

% \begin{solution}
Your solution goes here
\end{solution}

\begin{problem}(25 points)[Boneh-Shoup Exercise 2.10 from Section 2.6]
Let $\Scheme = (\Enc, \Dec)$ be a semantically
secure cipher defined over $(\K,\M, \C)$, where $\M = \C = \{0, 1\}^L$. 
Which of the following encryption algorithms yields a semantically secure scheme? 
Either give an attack or provide a security proof via an explicit reduction.

\begin{ppart}
    Let $\Scheme_{1}$ be a scheme s.t. $\Enc_1(k,m) := 0 || \Enc(k,m)$

    \begin{solution}
Your solution goes here
\end{solution}
\end{ppart}

\begin{ppart}
    Let $\Scheme_2$ be a scheme s.t. $\Enc_2(k,m) := \Enc(k,m) || \parity(m)$, where $\parity$ of a binary string refers to the number of 1 bits (equivalently, the exclusive-or of all the bits) in the string.

    \begin{solution}
Your solution goes here
\end{solution}
\end{ppart}


\end{problem}


\end{document}
 
